\documentclass{report}

\usepackage{amsmath}
\usepackage{syntax}

\begin{document}

\title{Dickinson Language Reference}
\author {Vanessa McHale}
\maketitle

\tableofcontents

\section{Introduction}

Dickinson is a language for generative literature targeting English. This reference specifies the syntax and semantics of the language.

\section{Syntax}

\subsection{Lexical Structure}

Dickinson programs have the following lexical structure:

\begin{align*}
    comment &=: {\tt ; .}^*\$ \\
    identifier &=: {\tt [a-z][a-zA-Z0-9]}^* \\
    typeIdentifier &=: {\tt [A-Z][a-zA-Z0-9]}^* \\
    moduleIdentifier &=: (identifier.)^* identifier \\
    include &=: {\tt :include} \\
    def &=: {\tt :def} \\
    lambda &=: {\tt :lambda} \\
    tydecl &=: {\tt tydecl} \\
    arrow &=: (\rightarrow | {\tt ->}) \\
    probability &=: ({\tt [0-9]}^+| {\tt [0-9]}^+.{\tt [0-9]}^*)
\end{align*}

\subsection{Syntax Tree}

\setlength{\grammarparsep}{20pt plus 1pt minus 1pt}
\setlength{\grammarindent}{12em}

\begin{grammar}
<pattern> ::= "_"
          \alt <identifier>
          \alt <typeIdentifier>

<type> ::= "text"
\alt $\rightarrow$ <type> <type>
\alt (<type> (, <type>)*)
\alt <identifier>

<expression> ::= <string>
\alt ("let:" [(<identifier> <expression>)+] <expression>)
\alt ("bind:" [(<identifier> <expression>)+] <expression>)
\alt (<expression> (, <expression>)*)
\alt (":flatten" <expression>)
\alt (<expression> : <type>)
\alt <typeIdentifier>
\alt (":pick" <identifier>)
\alt (">" <expression>*)
\alt (":oneof" ("|" <expression>)+)
\alt (":branch" ("|" <probability> <expression>)+)
\end{grammar}

\end{document}
