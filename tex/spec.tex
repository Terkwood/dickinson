\documentclass{book}

\usepackage{amsmath}

\begin{document}

\title{Dickinson Language Reference}
\author {Vanessa McHale}
\maketitle

\tableofcontents

\section{Introduction}

Dickinson is a language for generative literature targeting English. This reference specifies the syntax and semantics of the language.

\section{Syntax}

\subsection{Lexical Structure}

Dickinson programs have the following lexical structure:

\begin{align*}
    comment &=: {\tt ; .}^*\$ \\
    identifier &=: {\tt [a-z][a-zA-Z0-9]}^* \\
    typeIdentifier &=: {\tt [A-Z][a-zA-Z0-9]}^* \\
    moduleIdentifier &=: (identifier.)^* identifier \\
    include &=: {\tt :include} \\
    let &=: {\tt :let} \\
    match &=: {\tt :match} \\
    branch &=: {\tt :branch} \\
    oneof &=: {\tt :oneof} \\
    def &=: {\tt :def} \\
    lambda &=: {\tt :lambda} \\
    flatten &=: {\tt :flatten} \\
    text &=: {\tt text} \\
    tydecl &=: {\tt tydecl} \\
    arrow &=: (\rightarrow | {\tt ->}) \\
    probability &=: ({\tt [0-9]}^+| {\tt [0-9]}^+.{\tt [0-9]}^*)
\end{align*}

\subsection{Syntax Tree}

\begin{align*}
    expression =: &( ({\tt :let} identifier expression) \\
               &| string \\
               &)
\end{align*}

\end{document}
